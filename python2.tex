\documentclass{beamer}
%
% Choose how your presentation looks.
%
% For more themes, color themes and font themes, see:
% http://deic.uab.es/~iblanes/beamer_gallery/index_by_theme.html
%
\mode<presentation>
{
	\usetheme{Warsaw}      % or try Darmstadt, Madrid, Warsaw, ...
	\usecolortheme{default} % or try albatross, beaver, crane, ...
	\usefonttheme{default}  % or try serif, structurebold, ...
	\setbeamertemplate{navigation symbols}{}
	\setbeamertemplate{caption}[numbered]
} 

\usepackage[english]{babel}
\usepackage[utf8]{inputenc}
\usepackage{amsthm}
\usepackage{amssymb}
\usepackage{amsmath}
\usepackage{verbatim}
\usepackage{hyperref}
\usepackage{listings}
\usepackage{color}


\newtheorem{mydef}{Defini\c{c}\~ao}[section]
\newtheorem{myex}{Exemplo}[section]
\newtheorem{teo}{Teorema}[section]
\newtheorem{lema}[section]{Lema}
\newtheorem{cor}[section]{Corolário}
\newtheorem{prop}[section]{Proposição}
\newtheorem{usos}[section]{Usos}



\title[Python2]{Introdução à computação científica em economia e finanças em Python}
\author{Nelson Seixas dos Santos}
\institute{Faculdade de Ciências Econômicas\\Universidade Federal do Rio Grande do Sul}
\date{\today}


\begin{document}
	
	\begin{frame}
		\titlepage
	\end{frame}
	
	% Uncomment these lines for an automatically generated outline.
	\begin{frame}{Sumário}
		\tableofcontents
	\end{frame}
	
	
	
	\section{Introdução}
	
	
	\section{A biblioteca scipy.optimize}

	\begin{frame}{Equações não lineares}
		\begin{block}{Problema}
			Determinar a raiz da equação f(x) = 0 onde f é uma função não linear
		\end{block}
		\begin{block}{Soluções}
			\begin{enumerate}
				\item método da bissecção
				\item método da iteração linear
				\item método de Newton
				\item método das secantes
				\item método Regula Falsi
			\end{enumerate}
		\end{block}
	\end{frame}
	\begin{frame}{A função root}
		
	\end{frame}
	
	\section{A biblioteca scipy.linalg}
	
	\begin{frame}{Sistemas lineares}
		content...
	\end{frame}
	
	\begin{frame}{As funções inv e dot}
		Note que o sistema linear \begin{math}
			A.x = b
		\end{math}
		tem solução dada por \begin{math}
			x = A^{-1}.b
		\end{math}
		Portanto, a solução é obtida usando as funções inversa e multiplicação matricial.  Em Python: scipy.linalg.inv(A).dot(b)
	\end{frame}
		
	\begin{frame}{Autovetores e autovalores}
		Os autovalores de um operador linear (matriz quadrada) A são as raízes do polinômio característico.  Formalmente:
		\begin{equation*}
			|A - \lambda.I| = 0
		\end{equation*}
		onde I é a matriz identidade de mesma ordem que A.
		
		\begin{block}{A função scipy.linalg.elg}
			Esta função retorna diretamente os autovalores e autovetores de A.
		\end{block}
		
	\end{frame}
	
	
	\begin{frame}{Mínimos Quadrados Ordinários}
		scipy.linalg.lstsq
	\end{frame}
	
	\section{Interpolação polinomial}
	
	\begin{frame}{Interpolação polinomial}
		\begin{block}{Problema}
			Determinar o polinômio P(x) que passa pelos n+1 pontos $(x_0,y_0), (x_1,y_1), (x_2,y_2), ..., (x_{n-1},y_{n-1}), (x_n,y_n)$.
		\end{block}
	\end{frame}
	
	\begin{frame}{Solução}
		Para resolver o problema posto, lembre que a equação geral de um polinômio de n-ésimo grau é dada por:
		
		\begin{equation*}
		P(x)=a_0 + a_1 . x + a_2 . x^2 + ... + a_{n-1} . x^{n-1} + a_n . x^n
		\end{equation*}
		
		Como a equação acima tem  n+1 coeficientes, basta-nos  resolver o sistema linear a seguir:
		
		\begin{enumerate}
			\item $y_0 = P(x_0)$
			\item $y_1 = P(x_1)$
			\item ...
			\item $y_{n-1} = P(x_{n-1})$
			\item $y_n = P(x_n)$
		\end{enumerate}
	\end{frame}
	
	\begin{frame}{Exemplo}
		Encontre o polinômio que passa pelos pontos abaixo:
		\begin{enumerate}
			\item $(x_0,y_0)=(1,2)$
			\item $(x_1,y_1)=(3,4)$
			\item $(x_2,y_2)=(5,6)$
		\end{enumerate}
	\end{frame}
	
	
	
	
	\begin{frame}{Solução}
		\begin{equation}\label{eqquad}
		y=a.x^2+b.x+c
		\end{equation}
		Logo, substituindo as coordenadas dos pontos dados na equação (\ref{eqquad}) acima, temos:
		\begin{displaymath}\left\{
		\begin{array}{lr}
		a+b+c & = 2 \\
		9a+3b+c & = 4 \\
		25a+5b+c & = 6
		\end{array}
		\right.
		\end{displaymath}
	\end{frame}
	
	\begin{frame}{Scipy}
		Alternativamente, podemos usar a função de interpolação do pacote Python Scipy
	\end{frame}
	
	\begin{frame}{Exemplo: determinação da estrutura a termo das taxas de juros de uma economia}
		\begin{block}{Problema}
			Determinar o polinômio que que liga os pontos \begin{math}
			(t-t, r_{t,T})
			\end{math} no plano tempo-taxa.
		\end{block}
	\end{frame}
	
	
	
	
	
	
	
	\section{Exercícios}
	
	
	
	\begin{frame}{Menezes (2014, pp.71, 77)}
		\begin{enumerate}
			\item Faça um programa que peça dois números inteiros e ponha na tela a soma deles.
			\item Faça um programa que converta leia um valor em metros e o converta para milímetros
			\item Faça um programa que pergunte a velocidade do carro do usuários.  Caso ultrapasse 80km/h, exiba uma mensagem dizendo que o usuário foi multado.  Nesse caso, exiba o valor da multa, cobrando R \$ 5,00 por km acima de 80km/h.
		\end{enumerate}
	\end{frame}
	
	
	
	
	
	
	\begin{frame}{Outros exercícios}
		\begin{enumerate}
			\item Faça um programa que peça os coeficiente a, b e c de um polinômio de segundo grau e determine as raízes deste polinômio.
			\item Faça um programa que converta leia um valor em metros e o converta para milímetros
			\item Faça um programa que pergunte a velocidade do carro do usuário.  Caso ultrapasse 120km/h, exiba uma mensagem dizendo que o usuário foi multado.  Nesse caso, exiba o valor da multa, cobrando R \$ 5,00 por km acima de 120km/h.
			
		\end{enumerate}
		
	\end{frame}
	
	\section{Referências}
	\begin{frame}{Referências}
		\begin{itemize}
			\item \href{http://www.cursou.com.br/informatica/curso-de-phyton/}{Curso de Python}
			\item \href{https://www.python.org/}{Python Software Foundation}
			\item \href{https://www.packtpub.com/application-development/python-finance}{Python for Finance}
			\item \href{http://quant-econ.net/}{Quantitative Economics}
			\item \textbf{MENEZES}, Nilo Ney C. \textit{Introdução à programação com PYTHON: algoritmos e lógica de programação para iniciantes, 2a. Ed.}  São Paulo: NOVATEC, 2014
			\item \textbf{FOROUZAN}, B. e \textbf{MOSHARRAF}, F. \textit{Fundamentos da Ciência da Computação. Traduçao da 2a. ed. internacional}. São Paulo: CENGAGE Learning, 2011.
			\item \href{http://www.sciencedirect.com/science/article/pii/S0022053114001616}{Simpósio de Ciência da computação e Teoria Econômica}
		\end{itemize}
	\end{frame}
	
	
	
	
	
\end{document}
